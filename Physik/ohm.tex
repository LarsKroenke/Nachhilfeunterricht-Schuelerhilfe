\section*{Physik – Das Ohmsche Gesetz}

\textbf{Einführung:}

Das Ohmsche Gesetz beschreibt den Zusammenhang zwischen Spannung ($U$), Stromstärke ($I$) und Widerstand ($R$):

\begin{center}
$U = R \cdot I$
\end{center}

Dabei gilt:
\begin{itemize}
    \item $U$ = Spannung in Volt (V)
    \item $I$ = Stromstärke in Ampere (A)
    \item $R$ = Widerstand in Ohm ($\Omega$)
\end{itemize}

\vspace{0.5em}

\textbf{1. Ergänze die Tabelle (rechne mit $U = R \cdot I$):}

\begin{center}
\begin{tabular}{|c|c|c|c|}
\hline
Aufgabe & Spannung $U$ (V) & Stromstärke $I$ (A) & Widerstand $R$ ($\Omega$) \\
\hline
 a) & 12 & 2 & \\
 b) & 9 & & 3 \\
 c) & & 0{,}5 & 20 \\
 d) & 24 & 4 & \\
 e) & & 1{,}2 & 10 \\
\hline
\end{tabular}
\end{center}

\vspace{0.5em}

\textbf{2. Berechne:}
\begin{multicols}{2}
\begin{enumerate}[a)]
    \item Wie groß ist der Widerstand, wenn $U = 15\,\mathrm{V}$ und $I = 3\,\mathrm{A}$?
    \item Wie groß ist die Spannung, wenn $R = 8\,\Omega$ und $I = 0{,}5\,\mathrm{A}$?
    \item Wie groß ist die Stromstärke, wenn $U = 18\,\mathrm{V}$ und $R = 6\,\Omega$?
    \item Wie groß ist der Widerstand, wenn $U = 5\,\mathrm{V}$ und $I = 0{,}25\,\mathrm{A}$?
\end{enumerate}
\end{multicols}

\vspace{0.5em}

\textbf{3. Sachaufgaben:}
\begin{enumerate}[a)]
    \item Eine Glühlampe benötigt eine Spannung von $6\,\mathrm{V}$ und hat einen Widerstand von $12\,\Omega$. Wie groß ist die Stromstärke?
    \item Ein Föhn hat einen Widerstand von $40\,\Omega$ und wird mit $230\,\mathrm{V}$ betrieben. Wie groß ist der Strom?
    \item Ein Draht hat einen Widerstand von $5\,\Omega$. Wie groß ist die Spannung, wenn $2\,\mathrm{A}$ fließen?
\end{enumerate}

\vspace{0.5em}

\textbf{4. Knobelaufgabe:}

Ein Gerät darf höchstens $0{,}5\,\mathrm{A}$ Strom aufnehmen. Es wird an eine Spannung von $230\,\mathrm{V}$ angeschlossen. Wie groß muss der Widerstand mindestens sein?

\vspace{0.5em}

\textbf{5. Erkläre:}

\begin{enumerate}[a)]
    \item Was passiert mit der Stromstärke, wenn der Widerstand größer wird (bei gleicher Spannung)?
    \item Was passiert mit der Spannung, wenn der Strom steigt (bei gleichem Widerstand)?
\end{enumerate}

\textbf{Viel Erfolg!}