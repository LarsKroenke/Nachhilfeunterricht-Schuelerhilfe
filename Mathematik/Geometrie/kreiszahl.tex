
% Aufgabenblatt: Rechnen mit der Kreiszahl $\pi$
% Thema: Kreiszahl $\pi$ – Umfang, Flächeninhalt, Anwendungen
% Niveau: 9. Klasse
% Autor: [Dein Name]
% Datum: [Aktuelles Datum]

\section*{Aufgabenblatt: Rechnen mit der Kreiszahl $\pi$}

\textbf{Hinweis:} Verwende für $\pi$ den Näherungswert $\pi \approx 3{,}14$ oder den Taschenrechner.

\subsection*{Teil 1: Grundaufgaben}
\begin{enumerate}
    \item Berechne den Umfang eines Kreises mit dem Radius $r = 5~\text{cm}$.
    \item Berechne den Flächeninhalt eines Kreises mit dem Durchmesser $d = 10~\text{cm}$.
    \item Ein Kreis hat den Umfang $U = 31{,}4~\text{cm}$. Wie groß ist der Radius?
    \item Ein Kreis hat den Flächeninhalt $A = 78{,}5~\text{cm}^2$. Wie groß ist der Durchmesser?
    \item Ein Rad hat einen Durchmesser von $60~\text{cm}$. Wie weit rollt es nach 10 Umdrehungen?
\end{enumerate}

\subsection*{Teil 2: Anwendungsaufgaben aus Technik und Ingenieurwissenschaften}
\begin{enumerate}
    \setcounter{enumi}{5}
    \item Eine runde Metallplatte hat einen Radius von $8~\text{cm}$. Wie groß ist die Fläche, die lackiert werden muss?
    \item Ein Zylinder (z.B. eine Getränkedose) hat einen Durchmesser von $6~\text{cm}$ und eine Höhe von $12~\text{cm}$. Berechne die Grundfläche und das Volumen.
    \item Ein Zahnrad hat $r = 4~\text{cm}$. Wie lang ist der Weg, den ein Punkt am Rand bei einer vollen Umdrehung zurücklegt?
    \item Ein Rohr hat einen Außendurchmesser von $10~\text{cm}$ und einen Innendurchmesser von $8~\text{cm}$. Berechne die Querschnittsfläche des Rohres.
    \item Ein Draht wird zu einem Kreis mit $r = 20~\text{cm}$ gebogen. Wie lang ist der Draht?
\end{enumerate}

\subsection*{Teil 3: Knobel- und Sachaufgaben}
\begin{enumerate}
    \setcounter{enumi}{10}
    \item Ein rechteckiges Blech ($30~\text{cm} \times 20~\text{cm}$) soll in der Mitte ein rundes Loch mit $d = 10~\text{cm}$ erhalten. Wie groß ist die verbleibende Fläche?
    \item Ein Kreissektor hat einen Radius von $6~\text{cm}$ und einen Mittelpunktswinkel von $90^\circ$. Berechne die Länge des Kreisbogens und die Fläche des Sektors.
    \item Ein Autoreifen hat einen Durchmesser von $65~\text{cm}$. Wie oft dreht er sich auf einer Strecke von $1~\text{km}$?
    \item Ein runder Gartenteich hat einen Durchmesser von $4~\text{m}$. Wie viel Quadratmeter Folie werden mindestens benötigt?
    \item Ein Ingenieur konstruiert eine runde Platte mit $A = 314~\text{cm}^2$. Wie groß ist der Radius?
\end{enumerate}

\vspace{1cm}
\textbf{Viel Erfolg beim Lösen der Aufgaben!}
