\section{Wichtige Formeln}
\begin{itemize}
    \item \textbf{Würfel:} \quad Oberfläche: \( O = 6 \cdot a^2 \)
    \item \textbf{Quadratische Pyramide:} \quad Oberfläche: \( O = a^2 + 2 \cdot a \cdot h_s \)\\
    \quad (\( a \): Grundkante, \( h_s \): Höhe einer Seitenfläche)
    \item \textbf{Kugel:} \quad Oberfläche: \( O = 4 \cdot \pi \cdot r^2 \)
\end{itemize}

\vspace{1em}

\section{Teil A – Würfel}

\begin{enumerate}
    \item Berechne die Oberfläche der Würfel mit folgenden Kantenlängen:
    \begin{enumerate}
        \item \( a = 3\,\text{cm} \)
        \item \( a = 5\,\text{cm} \)
        \item \( a = 12\,\text{cm} \)
    \end{enumerate}

    \item Ein Würfel hat eine Oberfläche von \( 150\,\text{cm}^2 \). Berechne die Kantenlänge.

    \item Wie verändert sich die Oberfläche, wenn man die Kantenlänge verdoppelt?
\end{enumerate}

\vspace{1em}

\section{Teil B – Quadratische Pyramide}

\begin{enumerate}
    \setcounter{enumi}{3}
    \item Berechne die Oberfläche der Pyramiden mit:
    \begin{enumerate}
        \item \( a = 6\,\text{cm} \), \( h_s = 5\,\text{cm} \)
        \item \( a = 10\,\text{cm} \), \( h_s = 8\,\text{cm} \)
    \end{enumerate}

    \item Eine quadratische Pyramide hat eine Grundkante von \( 8\,\text{cm} \) und eine Oberfläche von \( 192\,\text{cm}^2 \). Berechne die Höhe einer Seitenfläche \( h_s \).

    \item Eine Pyramide wird auf einen Würfel gestellt (Kantenlänge \( 6\,\text{cm} \), \( h_s = 5\,\text{cm} \)). Berechne die gesamte Oberfläche des zusammengesetzten Körpers (ohne Innenflächen).
\end{enumerate}

\vspace{1em}

\section{Teil C – Kugel}

\begin{enumerate}
    \setcounter{enumi}{6}
    \item Berechne die Oberfläche folgender Kugeln:
    \begin{enumerate}
        \item \( r = 2\,\text{cm} \)
        \item \( r = 4{,}5\,\text{cm} \)
        \item \( d = 10\,\text{cm} \) \quad \textit{(Tipp: \( r = \frac{d}{2} \))}
    \end{enumerate}

    \item Eine Kugel hat eine Oberfläche von \( 314{,}16\,\text{cm}^2 \). Berechne den Radius (\( \pi \approx 3{,}14 \)).

    \item Was passiert mit der Oberfläche, wenn der Radius verdreifacht wird? Erkläre mathematisch.
\end{enumerate}

\vspace{1em}

\section{Teil D – Anwendungen und Sachaufgabe}

\begin{enumerate}
    \setcounter{enumi}{9}
    \item Ein Dekoobjekt besteht aus einer Holzkugel (\( r = 5\,\text{cm} \)), auf die eine goldene Pyramide gesetzt wird (\( a = 6\,\text{cm} \), \( h_s = 4{,}5\,\text{cm} \)).\\
    Berechne die gesamte Oberfläche, die bemalt werden muss.

    \item Ein Würfel hat dieselbe Oberfläche wie eine Kugel. Der Würfel hat die Kantenlänge \( a = 6\,\text{cm} \). Berechne den Radius der Kugel.
\end{enumerate}