\section*{Lernzielkontrolle: Lineare Funktionen (8. Klasse Gymnasium)}

\textbf{Name:} \underline{\hspace{5cm}} \hfill \textbf{Datum:} \underline{\hspace{3cm}}

\vspace{1em}

\textbf{Hinweis:} Bearbeite alle Aufgaben sorgfältig. Rechenschritte und Begründungen sind anzugeben. Taschenrechner sind nicht erlaubt.

\section*{1. Begriffe und Grundlagen}
\begin{enumerate}[label=\alph*)]
  \item Was ist eine lineare Funktion? Gib die allgemeine Funktionsgleichung an.\\[2em]
  \item Was bedeuten die Parameter $m$ und $b$ in der Gleichung $f(x) = mx + b$?\\[2em]
\end{enumerate}

\section*{2. Funktionsgleichungen und Graphen}
\begin{enumerate}[label=\alph*)]
  \item Zeichne den Graphen der Funktion $f(x) = 2x - 1$ für $x \in [-2, 3]$ in ein Koordinatensystem.\\[8em]
  \item Gib die Funktionsgleichung einer Geraden an, die durch die Punkte $A(0|2)$ und $B(2|6)$ verläuft.\\[3em]
  \item Bestimme die Nullstelle der Funktion $g(x) = -3x + 6$.\\[2em]
\end{enumerate}
\newpage
\section*{3. Steigung und y-Achsenabschnitt}
\begin{enumerate}[label=\alph*)]
  \item Welche Steigung hat die Gerade $h(x) = -\frac{1}{2}x + 4$?\\[2em]
  \item Wie lautet der y-Achsenabschnitt der Funktion $k(x) = 5x - 3$?\\[2em]
\end{enumerate}

\section*{4. Anwendungsaufgaben}
\begin{enumerate}[label=\alph*)]
  \item Ein Taxiunternehmen verlangt eine Grundgebühr von 3\,€ und pro Kilometer 1,80\,€. Stelle die Kostenfunktion $C(x)$ auf und berechne die Kosten für 12 km.\\[4em]
  \item Ein Wasserstand sinkt pro Stunde um 2,5 cm. Zu Beginn sind 120 cm im Behälter. Stelle eine Funktionsgleichung auf und berechne, nach wie vielen Stunden der Behälter leer ist.\\[4em]
\end{enumerate}

\section*{5. Sachaufgabe mit Skizze}
\textit{(Zeichne eine Skizze!)}\\
Ein Baum wächst jedes Jahr um 35 cm. Zu Beginn der Messung ist er 1,2 m hoch. Gib die Funktionsgleichung an und berechne, wie hoch der Baum nach 8 Jahren ist.

\vspace{2em}

\textbf{Viel Erfolg!}