% Aufgabenblatt: Lineare Funktionen – 8. Klasse
% Thema: Lineare Funktionen und deren Eigenschaften

\section*{Lineare Funktionen – 8. Klasse}

\subsection*{1. Grundlagen}
\begin{enumerate}
    \item Was ist eine lineare Funktion? Gib die allgemeine Funktionsgleichung an.
    \item Erkläre die Bedeutung der Parameter $m$ und $b$ in der Gleichung $y = mx + b$.
\end{enumerate}

\subsection*{2. Funktionsgleichungen aufstellen}
\begin{enumerate}
    \item Stelle die Funktionsgleichung der Geraden auf, die durch die Punkte $A(0|2)$ und $B(3|8)$ verläuft.
    \item Eine Gerade hat die Steigung $m = -2$ und schneidet die $y$-Achse bei $b = 5$. Gib die Funktionsgleichung an.
    \item Bestimme die Steigung und den $y$-Achsenabschnitt der Funktion $f(x) = -3x + 4$.
\end{enumerate}

\subsection*{3. Graphen zeichnen}
\begin{enumerate}
    \item Zeichne den Graphen der Funktion $f(x) = 2x - 1$ für $x \in [-2, 3]$.
    \item Zeichne die Gerade $g(x) = -x + 2$ im selben Koordinatensystem.
    \item Bestimme die Schnittpunkte der beiden Geraden.
\end{enumerate}

\subsection*{4. Anwendungsaufgaben}
\begin{enumerate}
    \item Ein Taxiunternehmen verlangt eine Grundgebühr von 3 € und pro Kilometer 1,50 €. Stelle die Kostenfunktion $K(x)$ auf und berechne die Kosten für 8 km.
    \item Ein Handyvertrag kostet monatlich 10 € Grundgebühr und 0,05 € pro SMS. Stelle die Kostenfunktion $C(x)$ auf und berechne die Kosten für 150 SMS im Monat.
\end{enumerate}

\subsection*{5. Textaufgaben}
\begin{enumerate}
    \item Ein Bäcker verkauft Brote für 2 € pro Stück. Er hat täglich Fixkosten von 40 €. Stelle die Gewinnfunktion $G(x)$ auf, wenn er $x$ Brote verkauft und jedes Brot für 2 € verkauft wird. Wie viele Brote muss er mindestens verkaufen, um Gewinn zu machen?
    \item Ein Auto verbraucht pro 100 km 6 Liter Benzin. Stelle eine Funktion auf, die den Benzinverbrauch $B(x)$ in Litern in Abhängigkeit von der gefahrenen Strecke $x$ in km angibt. Wie viel Benzin wird für 350 km benötigt?
\end{enumerate}

\vspace{1cm}
\textbf{Viel Erfolg!}
