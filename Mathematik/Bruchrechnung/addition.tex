\section*{Bruchrechnung – Addition von Brüchen}

\textbf{Einführung:}

Um Brüche zu addieren, müssen sie denselben Nenner (Hauptnenner) haben. Falls nötig, erweitere die Brüche zuerst. Addiere dann die Zähler und schreibe das Ergebnis über den gemeinsamen Nenner. Kürze das Ergebnis, wenn möglich.

\textbf{Beispiel:}

$\frac{1}{4} + \frac{1}{2} = \frac{1}{4} + \frac{2}{4} = \frac{3}{4}$

\section*{Aufgaben}

\textbf{1. Addiere die Brüche (gleicher Nenner):}
\begin{multicols}{2}
\begin{enumerate}[label=\alph*)]
  \item $\displaystyle \frac{2}{7} + \frac{3}{7}$
  \item $\displaystyle \frac{5}{9} + \frac{1}{9}$
  \item $\displaystyle \frac{4}{11} + \frac{2}{11}$
  \item $\displaystyle \frac{7}{8} + \frac{1}{8}$
\end{enumerate}
\end{multicols}

\textbf{2. Addiere die Brüche (verschiedene Nenner, Hauptnenner finden):}
\begin{multicols}{2}
\begin{enumerate}[label=\alph*)]
  \item $\displaystyle \frac{1}{3} + \frac{1}{6}$
  \item $\displaystyle \frac{2}{5} + \frac{1}{10}$
  \item $\displaystyle \frac{3}{4} + \frac{1}{8}$
  \item $\displaystyle \frac{5}{12} + \frac{1}{3}$
\end{enumerate}
\end{multicols}

\textbf{3. Addiere und kürze das Ergebnis, wenn möglich:}
\begin{multicols}{2}
\begin{enumerate}[label=\alph*)]
  \item $\displaystyle \frac{2}{6} + \frac{1}{3}$
  \item $\displaystyle \frac{3}{9} + \frac{2}{6}$
  \item $\displaystyle \frac{4}{10} + \frac{1}{5}$
  \item $\displaystyle \frac{6}{8} + \frac{1}{4}$
\end{enumerate}
\end{multicols}

\textbf{4. Addiere gemischte Zahlen:}
\begin{enumerate}[label=\alph*)]
  \item $\displaystyle 1\frac{1}{2} + 2\frac{1}{4}$
  \item $\displaystyle 3\frac{2}{3} + 1\frac{1}{6}$
  \item $\displaystyle 2\frac{3}{5} + 1\frac{2}{5}$
\end{enumerate}
\newpage
\textbf{5. Sachaufgaben:}
\begin{enumerate}[label=\alph*)]
  \item Anna backt einen Kuchen und verwendet $\displaystyle \frac{1}{2}$ Liter Milch und $\displaystyle \frac{1}{4}$ Liter Sahne. Wie viel Flüssigkeit verwendet sie insgesamt?
  \item Ein Schüler läuft morgens $\displaystyle \frac{3}{8}$ km und nachmittags $\displaystyle \frac{5}{8}$ km. Wie viele Kilometer läuft er insgesamt?
  \item In einer Klasse sind $\displaystyle \frac{2}{5}$ der Schüler Mädchen und $\displaystyle \frac{1}{4}$ Jungen mit Brille. Wie viele Schüler haben insgesamt eine Brille? (Brüche addieren, Ergebnis als Bruch)
\end{enumerate}

\textbf{Viel Erfolg!}
