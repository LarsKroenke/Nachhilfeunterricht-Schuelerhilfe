% --- Aufgabenblatt: Gemischte Brüche (7. Klasse) ---

\section*{Gemischte Brüche – Aufgabenblatt}

\textbf{Hinweis:} Ein gemischter Bruch besteht aus einer ganzen Zahl und einem echten Bruch, z.B. $2\dfrac{1}{3}$. Man kann gemischte Brüche in unechte Brüche umwandeln und umgekehrt.

\subsection*{1. Schreibe als gemischten Bruch}
\begin{enumerate}[a)]
    \item $\dfrac{7}{2}$
    \item $\dfrac{13}{4}$
    \item $\dfrac{22}{5}$
    \item $\dfrac{17}{3}$
    \item $\dfrac{25}{6}$
\end{enumerate}

\subsection*{2. Schreibe als unechten Bruch}
\begin{enumerate}[a)]
    \item $2\dfrac{1}{4}$
    \item $3\dfrac{2}{5}$
    \item $1\dfrac{3}{7}$
    \item $4\dfrac{2}{3}$
    \item $5\dfrac{1}{6}$
\end{enumerate}

\subsection*{3. Addiere oder subtrahiere die gemischten Brüche}
\begin{enumerate}[a)]
    \item $1\dfrac{1}{2} + 2\dfrac{1}{2}$
    \item $3\dfrac{1}{3} - 1\dfrac{2}{3}$
    \item $2\dfrac{3}{4} + 1\dfrac{1}{4}$
    \item $4\dfrac{1}{5} - 2\dfrac{2}{5}$
    \item $5\dfrac{1}{6} + 2\dfrac{5}{6}$
\end{enumerate}

\subsection*{4. Multipliziere und dividiere gemischte Brüche (als unechte Brüche rechnen)}
\begin{enumerate}[a)]
    \item $1\dfrac{1}{2} \cdot 2\dfrac{2}{3}$
    \item $3\dfrac{1}{4} : 1\dfrac{1}{2}$
    \item $2\dfrac{2}{5} \cdot 1\dfrac{3}{5}$
    \item $4\dfrac{1}{3} : 2\dfrac{2}{3}$
\end{enumerate}

\subsection*{5. Textaufgaben}
\begin{enumerate}[a)]
    \item Anna läuft jeden Tag $2\dfrac{1}{2}$ km zur Schule und zurück. Wie viele Kilometer läuft sie in 5 Tagen?
    \item Ein Kuchenrezept benötigt $1\dfrac{3}{4}$ Liter Milch. Wie viel Milch braucht man für 3 Kuchen?
    \item Ein Brett ist $4\dfrac{1}{2}$ Meter lang. Es soll in Stücke von je $1\dfrac{1}{2}$ Meter gesägt werden. Wie viele Stücke erhält man?
\end{enumerate}

\vspace{1em}
\textbf{Tipp:} Wandle gemischte Brüche für Rechnungen immer zuerst in unechte Brüche um!