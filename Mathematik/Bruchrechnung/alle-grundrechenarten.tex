% --- Aufgabenblatt: Bruchrechnung – Alle Grundrechenarten ---

\section*{Bruchrechnung – Alle Grundrechenarten}

\subsection*{1. Addition und Subtraktion von Brüchen}
\begin{tabular}{p{0.45\textwidth} p{0.45\textwidth}}
 a) $\frac{2}{5} + \frac{1}{5} =$ & b) $\frac{3}{4} - \frac{1}{8} =$ \\
 c) $\frac{5}{6} + \frac{1}{3} =$ & d) $\frac{7}{10} - \frac{2}{5} =$ \\
 e) $\frac{2}{3} + \frac{3}{7} =$ & f) $\frac{9}{12} - \frac{1}{6} =$ \\
\end{tabular}

\vspace{0.5cm}

\subsection*{2. Multiplikation von Brüchen}
\begin{tabular}{p{0.45\textwidth} p{0.45\textwidth}}
 a) $\frac{2}{3} \cdot \frac{3}{4} =$ & b) $\frac{5}{8} \cdot \frac{4}{7} =$ \\
 c) $\frac{7}{9} \cdot \frac{2}{5} =$ & d) $\frac{3}{10} \cdot \frac{5}{6} =$ \\
 e) $\frac{1}{2} \cdot \frac{1}{3} =$ & \\
\end{tabular}

\vspace{0.5cm}

\subsection*{3. Division von Brüchen}
\begin{tabular}{p{0.45\textwidth} p{0.45\textwidth}}
 a) $\frac{3}{4} : \frac{2}{5} =$ & b) $\frac{7}{8} : \frac{1}{4} =$ \\
 c) $\frac{5}{6} : \frac{2}{3} =$ & d) $\frac{9}{10} : \frac{3}{5} =$ \\
 e) $\frac{4}{7} : \frac{2}{3} =$ & \\
\end{tabular}

\vspace{0.5cm}

\subsection*{4. Gemischte Aufgaben}
\begin{tabular}{p{0.45\textwidth} p{0.45\textwidth}}
 a) $\frac{2}{3} + \frac{1}{4} - \frac{1}{6} =$ & b) $\frac{5}{8} \cdot \frac{4}{5} + \frac{1}{2} =$ \\
 c) $\frac{7}{9} - \frac{2}{3} \cdot \frac{3}{7} =$ & d) $\frac{3}{5} : \frac{6}{25} + \frac{1}{2} =$ \\
 e) $\frac{4}{7} + \frac{2}{3} - \frac{1}{2} =$ & \\
\end{tabular}

\vspace{0.5cm}

\subsection*{5. Textaufgaben}
\begin{enumerate}[a)]
    \item Ein Kuchen wird in 8 gleich große Stücke geteilt. Anna isst 3 Stücke, Ben isst 2 Stücke. Wie viel des Kuchens bleibt übrig?
    \item Ein Seil ist $\frac{3}{4}$ Meter lang. Es wird in Stücke von $\frac{1}{8}$ Meter geschnitten. Wie viele Stücke erhält man?
    \item Von einer Tafel Schokolade isst Lisa $\frac{2}{5}$, Tom isst $\frac{1}{4}$. Wie viel bleibt übrig?
    \item Ein Behälter ist zu $\frac{5}{6}$ mit Wasser gefüllt. Es werden $\frac{1}{2}$ Liter entnommen. Wie viel Bruchteil des vollen Behälters bleibt, wenn der Behälter 3 Liter fasst?
    \item Ein Rezept benötigt $\frac{2}{3}$ Liter Milch. Du hast nur einen Messbecher für $\frac{1}{6}$ Liter. Wie oft musst du den Messbecher füllen?
\end{enumerate}

% --- Ende des Aufgabenblatts ---
