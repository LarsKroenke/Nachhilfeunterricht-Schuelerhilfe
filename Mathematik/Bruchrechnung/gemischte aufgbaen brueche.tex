
% Aufgabenblatt: Gemischte Aufgaben zu Brüchen
% Thema: Bruchrechnung (ggT, Kürzen, Addition, Subtraktion, Multiplikation, Division)
% Autor: [Dein Name]
% Datum: [Aktuelles Datum]

\section*{Aufgabenblatt: Gemischte Aufgaben zu Brüchen}

\textbf{Hinweis:} Bearbeite die Aufgaben sorgfältig. Zwischenschritte und das Kürzen der Brüche sind ausdrücklich zu notieren.

\subsection*{Teil 1: ggT und Brüche kürzen}
\begin{enumerate}
    \item Kürze die folgenden Brüche so weit wie möglich:
    \begin{enumerate}[a)]
        \item $\frac{12}{18}$
        \item $\frac{45}{60}$
        \item $\frac{28}{42}$
        \item $\frac{81}{108}$
        \item $\frac{35}{49}$
    \end{enumerate}
    \item Bestimme den größten gemeinsamen Teiler (ggT) der folgenden Zahlenpaare:
    \begin{enumerate}[a)]
        \item 24 und 36
        \item 56 und 98
        \item 63 und 105
        \item 84 und 126
        \item 99 und 121
    \end{enumerate}
\end{enumerate}

\subsection*{Teil 2: Addition und Subtraktion von Brüchen}
\begin{enumerate}
    \setcounter{enumi}{2}
    \item Addiere oder subtrahiere und kürze das Ergebnis:
    \begin{enumerate}[a)]
        \item $\frac{2}{5} + \frac{1}{3}$
        \item $\frac{7}{8} - \frac{1}{4}$
        \item $\frac{5}{6} + \frac{1}{2}$
        \item $\frac{9}{10} - \frac{2}{5}$
        \item $\frac{3}{7} + \frac{2}{21}$
    \end{enumerate}
\end{enumerate}

\subsection*{Teil 3: Multiplikation und Division von Brüchen}
\begin{enumerate}
    \setcounter{enumi}{3}
    \item Multipliziere und kürze das Ergebnis:
    \begin{enumerate}[a)]
        \item $\frac{3}{4} \cdot \frac{2}{5}$
        \item $\frac{7}{9} \cdot \frac{3}{14}$
        \item $\frac{5}{8} \cdot \frac{4}{15}$
        \item $\frac{6}{11} \cdot \frac{11}{12}$
        \item $\frac{9}{10} \cdot \frac{5}{18}$
    \end{enumerate}
    \item Dividiere und kürze das Ergebnis:
    \begin{enumerate}[a)]
        \item $\frac{4}{7} : \frac{2}{3}$
        \item $\frac{5}{12} : \frac{10}{9}$
        \item $\frac{8}{15} : \frac{2}{5}$
        \item $\frac{3}{8} : \frac{9}{16}$
        \item $\frac{7}{18} : \frac{14}{27}$
    \end{enumerate}
\end{enumerate}

\subsection*{Teil 4: Gemischte Aufgaben}
\begin{enumerate}
    \setcounter{enumi}{5}
    \item Führe die angegebenen Rechenschritte aus und kürze jeweils das Ergebnis:
    \begin{enumerate}[a)]
        \item $\frac{2}{3} + \frac{1}{6} - \frac{1}{2}$
        \item $\frac{5}{8} \cdot \frac{4}{15} + \frac{1}{3}$
        \item $\frac{7}{9} - \frac{2}{3} : \frac{1}{6}$
        \item $\frac{3}{5} + \frac{2}{7} \cdot \frac{7}{8}$
        \item $\frac{9}{10} - \frac{1}{2} + \frac{3}{20}$
    \end{enumerate}
\end{enumerate}

\vspace{1cm}
\textbf{Viel Erfolg beim Lösen der Aufgaben!}
