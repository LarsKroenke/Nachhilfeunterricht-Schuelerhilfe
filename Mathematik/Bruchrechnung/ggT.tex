\section*{Bruchrechnung – Der größte gemeinsame Teiler (ggT)}

\textbf{Einführung:}

Der größte gemeinsame Teiler (ggT) zweier Zahlen ist die größte Zahl, durch die beide Zahlen ohne Rest teilbar sind. Der ggT ist wichtig beim Kürzen von Brüchen und beim Lösen von Aufgaben mit Teilbarkeiten.

\textbf{Beispiel:} Die Zahlen 18 und 24 haben die gemeinsamen Teiler 1, 2, 3, 6. Der größte davon ist 6, also ist $\mathrm{ggT}(18,24) = 6$.

\vspace{1em}

\section*{Aufgaben}

\textbf{1. Bestimme den ggT der folgenden Zahlenpaare:}
\begin{enumerate}[label=\alph*)]
  \item 12 und 30
  \item 21 und 49
  \item 36 und 60
  \item 45 und 120
  \item 56 und 98
\end{enumerate}

\textbf{2. Finde jeweils alle gemeinsamen Teiler und markiere den ggT:}
\begin{enumerate}[label=\alph*)]
  \item 16 und 40
  \item 27 und 63
  \item 81 und 54
\end{enumerate}

\textbf{3. Kürze die Brüche vollständig mit Hilfe des ggT:}
\begin{enumerate}[label=\alph*)]
  \item $\displaystyle{\frac{24}{36}}$
  \item $\displaystyle{\frac{35}{49}}$
  \item $\displaystyle{\frac{42}{56}}$
  \item $\displaystyle{\frac{90}{120}}$
\end{enumerate}
\newpage
\textbf{4. Anwendungsaufgabe:}

Ein Bäcker möchte 48 Brötchen und 60 Brezeln in möglichst große, gleich große Tüten ohne Reste verpacken. Wie viele Brötchen und Brezeln kommen in jede Tüte? Wie viele Tüten braucht er?