% --- Aufgabenblatt: Schriftliche Addition bis 1000 ---

\section*{Aufgabenblatt: Schriftliche Addition bis 1000}

\subsection*{1. Addiere im Kopf (Ergebnisse unter 1000)}
\begin{enumerate}[a)]
    \item $245 + 123 =$
    \item $378 + 211 =$
    \item $456 + 89 =$
    \item $512 + 387 =$
    \item $299 + 401 =$
    \item $650 + 149 =$
    \item $800 + 175 =$
    \item $999 + 1 =$
\end{enumerate}

\subsection*{2. Schriftliche Addition (Zahlen unter 1000)}
Addiere schriftlich. Notiere jeden Zwischenschritt.
\begin{enumerate}[a)]
    \item $234 + 567 =$
    \item $489 + 312 =$
    \item $678 + 215 =$
    \item $523 + 476 =$
    \item $345 + 654 =$
    \item $789 + 198 =$
    \item $407 + 592 =$
    \item $356 + 644 =$
\end{enumerate}

\subsection*{3. Drei Summanden}
Addiere schriftlich. Notiere alle Zwischenschritte.
\begin{enumerate}[a)]
    \item $123 + 456 + 321 =$
    \item $234 + 345 + 210 =$
    \item $150 + 275 + 575 =$
    \item $400 + 299 + 101 =$
    \item $512 + 123 + 365 =$
    \item $678 + 111 + 89 =$
\end{enumerate}

\subsection*{4. Sachaufgaben}
\begin{enumerate}[a)]
    \item Anna sammelt 245 Murmeln, Ben 378 und Clara 156. Wie viele Murmeln haben sie zusammen?
    \item Ein Bus fährt morgens 123 km, mittags 256 km und abends 198 km. Wie viele Kilometer fährt er insgesamt?
    \item In einer Klasse sind 234 Jungen, 267 Mädchen und 99 Gastschüler. Wie viele Schüler sind es insgesamt?
    \item Ein Laden verkauft am Montag 189, am Dienstag 276 und am Mittwoch 312 Brötchen. Wie viele Brötchen wurden verkauft?
\end{enumerate}

\subsection*{5. Knobelaufgaben}
\begin{enumerate}[a)]
    \item Finde zwei verschiedene Zahlen zwischen 100 und 900, deren Summe genau 1000 ergibt.
    \item Welche drei Zahlen zwischen 200 und 400 ergeben zusammen 999?
    \item Ergänze: $\underline{\hspace{1cm}} + 456 = 1000$
    \item Ergänze: $789 + \underline{\hspace{1cm}} = 1000$
\end{enumerate}

% --- Ende des Aufgabenblatts ---
