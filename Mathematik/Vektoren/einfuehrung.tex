% Einführung in die Vektorrechnung
% Arbeitsblatt für Schülerinnen und Schüler

\section*{Was ist ein Vektor?}
Ein Vektor ist eine gerichtete Größe. Er besitzt eine Richtung und eine Länge (Betrag). Vektoren werden oft als Pfeile dargestellt.

\subsection*{Beispiele für Vektoren:}
\begin{itemize}
  \item Eine Verschiebung auf einem Blatt Papier
  \item Geschwindigkeit (z.B. Windrichtung und -stärke)
  \item Kräfte
\end{itemize}

\section*{Darstellung von Vektoren}
Ein Vektor im Koordinatensystem wird meist so geschrieben:
\[
\vec{a} = \begin{pmatrix} 2 \\ 3 \end{pmatrix}
\]
Das bedeutet: 2 Einheiten nach rechts, 3 Einheiten nach oben.

\section*{Aufgaben}
\begin{enumerate}
  \item Zeichne den Vektor $\vec{a} = \begin{pmatrix} 3 \\ 1 \end{pmatrix}$ in ein Koordinatensystem.
  \item Berechne die Länge (Betrag) des Vektors $\vec{b} = \begin{pmatrix} 4 \\ 3 \end{pmatrix}$.
  \item Addiere die Vektoren $\vec{a} = \begin{pmatrix} 1 \\ 2 \end{pmatrix}$ und $\vec{b} = \begin{pmatrix} 2 \\ 5 \end{pmatrix}$.
  \item Subtrahiere $\vec{b}$ von $\vec{a}$: $\vec{a} - \vec{b}$.
\end{enumerate}

\section*{Hinweis}
Die Länge eines Vektors $\vec{v} = \begin{pmatrix} x \\ y \end{pmatrix}$ berechnet man mit:
\[
|\vec{v}| = \sqrt{x^2 + y^2}
\]

Viel Erfolg beim Bearbeiten des Arbeitsblatts!