% --- Aufgabenblatt: Äquivalenzumformungen mit Brüchen ---

\section*{Äquivalenzumformungen mit Brüchen}

\textbf{Hinweis:} Bei Äquivalenzumformungen darfst du beide Seiten einer Gleichung mit derselben Zahl (außer 0) multiplizieren oder dividieren sowie denselben Wert addieren oder subtrahieren.

\vspace{1em}

\subsection*{1. Löse die Gleichungen nach $x$ auf.}

\begin{enumerate}[a)]
    \item $\displaystyle \frac{x}{3} = 5$
    \item $\displaystyle \frac{x}{4} + 2 = 6$
    \item $\displaystyle \frac{x-1}{5} = 3$
    \item $\displaystyle 2 + \frac{x}{7} = 6$
    \item $\displaystyle \frac{x+2}{6} = 4$
    \item $\displaystyle \frac{2x}{9} = 8$
\end{enumerate}

\vspace{1em}

\subsection*{2. Bringe die Gleichungen in die Form $x = \ldots$ und löse.}

\begin{enumerate}[a)]
    \item $\displaystyle \frac{x}{2} - 3 = 5$
    \item $\displaystyle \frac{x+4}{3} = 7$
    \item $\displaystyle \frac{2x-1}{5} = 3$
    \item $\displaystyle \frac{x}{8} + 1 = 4$
    \item $\displaystyle \frac{x-5}{6} = 2$
    \item $\displaystyle \frac{3x}{4} = 9$
\end{enumerate}

\vspace{1em}

\subsection*{3. Löse die Gleichungen mit Brüchen auf beiden Seiten.}

\begin{enumerate}[a)]
    \item $\displaystyle \frac{x}{3} = \frac{4}{5}$
    \item $\displaystyle \frac{x-2}{6} = \frac{1}{2}$
    \item $\displaystyle \frac{2x}{7} = \frac{3}{4}$
    \item $\displaystyle \frac{x+1}{5} = \frac{7}{10}$
    \item $\displaystyle \frac{x}{8} + \frac{1}{2} = 2$
    \item $\displaystyle \frac{x-3}{4} = \frac{5}{2}$
\end{enumerate}

\vspace{1em}

\subsection*{4. Knobelaufgaben}

\begin{enumerate}[a)]
    \item Finde alle $x \in \mathbb{Z}$, für die gilt: $\displaystyle \frac{x}{4} = \frac{12}{x}$
    \item Für welche $x$ gilt: $\displaystyle \frac{x-2}{x+2} = \frac{3}{5}$?
    \item Löse: $\displaystyle \frac{2x+1}{3} = \frac{x-2}{2}$
    \item Löse: $\displaystyle \frac{x}{5} + \frac{x}{10} = 6$
\end{enumerate}

\vspace{1em}

\textbf{Tipp:} Multipliziere beide Seiten der Gleichung mit dem Hauptnenner, um die Brüche zu beseitigen!