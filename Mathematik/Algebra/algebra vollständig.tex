% --- Aufgabenblatt: Algebraische Tricks für die Mittelstufe ---

\section*{Algebra: Wichtige Tricks und Methoden}

\subsection*{1. Klammern auflösen und zusammenfassen}
\begin{enumerate}[a)]
    \item $2(x+3)$
    \item $5(a-4) + 2a$
    \item $3(x+2) - 2(x-1)$
    \item $4y - (2y + 5)$
    \item $2(x-3) + 3(x+1) - x$
\end{enumerate}

\subsection*{2. Ausklammern (Faktorisieren)}
\begin{enumerate}[a)]
    \item $6x + 12$
    \item $8a - 4b$
    \item $15y + 10$
    \item $9x - 3y$
    \item $12a + 18b$
\end{enumerate}

\subsection*{3. Distributivgesetz anwenden}
\begin{enumerate}[a)]
    \item $3(x+4)$
    \item $2(a-b)$
    \item $-5(y-2)$
    \item $4(x-3) + 2(x+1)$
\end{enumerate}

\subsection*{4. Binomische Formeln}
\textbf{A) Berechne:}
\begin{enumerate}[a)]
    \item $(a+b)^2$
    \item $(x-5)^2$
    \item $(2y+3)^2$
    \item $(a-b)^2$
    \item $(x+4)(x-4)$
\end{enumerate}

\subsection*{5. Terme vereinfachen}
\begin{enumerate}[a)]
    \item $2x + 3x - x$
    \item $4a - 2a + 7$
    \item $5y + 2y - 3y + 1$
    \item $3x^2 + 2x - x^2 + 4$
    \item $6a^2 - 2a^2 + a$
\end{enumerate}

\subsection*{6. Brüche mit Variablen}
\begin{enumerate}[a)]
    \item $\frac{2x}{3} + \frac{x}{6}$
    \item $\frac{a}{4} - \frac{2a}{8}$
    \item $\frac{3y}{5} + \frac{2y}{5}$
    \item $\frac{x+2}{3} - \frac{x}{3}$
    \item $\frac{4a}{7} + \frac{2}{7}$
\end{enumerate}

\subsection*{7. Gleichungen lösen}
\begin{enumerate}[a)]
    \item $2x + 3 = 11$
    \item $5a - 7 = 18$
    \item $3(x-2) = 9$
    \item $4y + 2 = 3y + 8$
    \item $2(x+1) = x + 7$
\end{enumerate}

\subsection*{8. Faktorisieren und Sonderfälle}
\begin{enumerate}[a)]
    \item $x^2 - 9$
    \item $a^2 - 4a + 4$
    \item $y^2 - 16$
    \item $x^2 + 6x + 9$
    \item $4a^2 - 25$
\end{enumerate}

\subsection*{9. Umformen und Einsetzen}
\begin{enumerate}[a)]
    \item Stelle $y$ nach $x$ um: $y = 2x + 5$
    \item Stelle $x$ nach $y$ um: $y = 3x - 4$
    \item Setze $x=2$ in $y = x^2 + 3x$ ein.
    \item Setze $a=5$ in $b = 2a - 7$ ein.
    \item Setze $y=4$ in $x = \frac{y-2}{3}$ ein.
\end{enumerate}

% \subsection*{10. Textaufgaben (algebraisch lösen)}
% \begin{enumerate}[a)]
%     \item Ein Vater ist dreimal so alt wie sein Sohn. Zusammen sind sie 48 Jahre alt. Wie alt sind beide?
%     \item Ein Rechteck ist doppelt so lang wie breit. Der Umfang beträgt 36 cm. Wie lang und wie breit ist das Rechteck?
%     \item Die Summe aus einer Zahl und ihrem Doppelten ist 27. Wie heißt die Zahl?
%     \item Ein Schüler kauft 3 Hefte und 2 Stifte für 8 €. Ein Heft kostet 2 €, wie viel kostet ein Stift?
%     \item Ein Dreieck hat einen Umfang von 24 cm. Zwei Seiten sind gleich lang, die dritte ist 6 cm kürzer. Wie lang sind die Seiten?
% \end{enumerate}

% \subsection*{11. Knobelaufgaben und Tricks}
% \begin{enumerate}[a)]
%     \item Finde zwei verschiedene Zahlen, deren Produkt gleich ihrer Summe ist.
%     \item Finde eine Zahl, die mit sich selbst multipliziert um 6 größer ist als das Doppelte dieser Zahl.
%     \item Für welche $x$ gilt: $x^2 = 2x$?
%     \item Finde alle $x$, für die $x^2 - 4x + 4 = 0$ gilt.
%     \item Finde ein Beispiel, wo das Distributivgesetz das Rechnen vereinfacht.
% \end{enumerate}
