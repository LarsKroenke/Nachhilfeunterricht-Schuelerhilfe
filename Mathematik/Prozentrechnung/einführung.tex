
% Arbeitsblatt: Einführung in die Prozentrechnung
% Thema: Prozentrechnung – Grundlagen, Erklärungen und Beispiele
% Autor: [Dein Name]
% Datum: [Aktuelles Datum]

\section*{Einführung in die Prozentrechnung}

\textbf{Was bedeutet Prozent?}

Das Wort "Prozent" kommt aus dem Lateinischen und bedeutet "von Hundert". Das Prozentzeichen \% steht also für einen Anteil von 100. Beispiel: 25\% bedeutet 25 von 100, also $\frac{25}{100}$.

\subsection*{Grundbegriffe der Prozentrechnung}
\begin{itemize}
    \item \textbf{Grundwert (G):} Das Ganze, auf das sich der Prozentsatz bezieht.
    \item \textbf{Prozentsatz (p):} Gibt an, wie viel Hundertstel vom Grundwert genommen werden (z.B. 15\%).
    \item \textbf{Prozentwert (W):} Der Anteil, der dem Prozentsatz vom Grundwert entspricht.
\end{itemize}

\textbf{Wichtige Formeln:}
\begin{align*}
    W &= \frac{p}{100} \cdot G \\
    p &= \frac{W}{G} \cdot 100 \\
    G &= \frac{W}{p} \cdot 100
\end{align*}


\subsection*{Beispiel 1: Prozentwert berechnen}
\textit{Frage:} Wie viel sind 30\% von 250~\euro?

\textbf{Lösung:}
\begin{align*}
    W &= \frac{30}{100} \cdot 250 = 0{,}3 \cdot 250 = 75~\euro
\end{align*}

\subsection*{Beispiel 2: Prozentsatz berechnen}
\textit{Frage:} Wie viel Prozent sind 45~\euro{} von 300~\euro{}?

\textbf{Lösung:}
\begin{align*}
    p &= \frac{45}{300} \cdot 100 = 0{,}15 \cdot 100 = 15\%
\end{align*}

\subsection*{Beispiel 3: Grundwert berechnen}
\textit{Frage:} 60~\euro{} sind 20\% eines Betrags. Wie groß ist der Grundwert?

\textbf{Lösung:}
\begin{align*}
    G &= \frac{60}{20} \cdot 100 = 3 \cdot 100 = 300~\euro
\end{align*}

\subsection*{Aufgaben zum Üben}
\begin{enumerate}
    \item Berechne 15\% von 320~\euro.
    \item Wie viel Prozent sind 48~g von 400~g?
    \item 72~km sind 12\% einer Strecke. Wie lang ist die gesamte Strecke?
    \item Ein Pullover kostet nach 25\% Rabatt noch 45~\euro. Wie teuer war er vorher?
    \item In einer Klasse sind 12 von 30 Schülern Mädchen. Wie viel Prozent sind das?
\end{enumerate}

\vspace{1cm}
\textbf{Tipp:} Schreibe immer auf, was gegeben und was gesucht ist. Nutze die Formeln und rechne sorgfältig.

\textbf{Viel Erfolg beim Üben!}
