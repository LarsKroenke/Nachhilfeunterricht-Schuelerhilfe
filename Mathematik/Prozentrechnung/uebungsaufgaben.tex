% Übungsaufgaben zur Prozentrechnung
% Arbeitsblatt für Schülerinnen und Schüler

\section*{Prozentrechnung – Übungsaufgaben}

\begin{enumerate}
  \item Wie viel sind 25\% von 200 Euro?
  \item Ein Pullover kostet nach einer Reduzierung von 20\% noch 48 Euro. Wie viel hat er vorher gekostet?
  \item In einer Klasse sind 28 Schüler, davon sind 9 Mädchen. Wie viel Prozent der Klasse sind Mädchen?
  \item Ein Handy wird um 15\% teurer und kostet jetzt 230 Euro. Wie viel hat es vorher gekostet?
  \item Ein Konto wird mit 2,5\% verzinst. Wie viel Zinsen erhält man für 800 Euro nach einem Jahr?
  \item Von 500 Schülern einer Schule nehmen 120 an einer AG teil. Wie viel Prozent sind das?
  \item Ein Fernseher kostet 600 Euro. Im Angebot gibt es 10\% Rabatt. Wie viel muss man bezahlen?
  \item Ein Preis steigt von 80 Euro auf 100 Euro. Um wie viel Prozent ist der Preis gestiegen?
  \item Ein Auto verliert im ersten Jahr 12\% seines Wertes. Wie viel ist ein Auto nach einem Jahr noch wert, wenn es neu 18.000 Euro gekostet hat?
  \item In einer Umfrage geben 42 von 70 Befragten an, regelmäßig Sport zu treiben. Wie viel Prozent sind das?
\end{enumerate}

Viel Erfolg beim Lösen der Aufgaben!