% --- Arbeitsblatt: Grundlagen der Stochastik (Oberstufenvorbereitung) ---

\section*{Grundlagen der Stochastik – Vorbereitung auf die Oberstufe}

\subsection*{1. Zufallsexperimente und Ereignisse}
\begin{enumerate}[a)]
    \item Erkläre den Unterschied zwischen Zufallsexperiment, Ergebnis und Ereignis an einem Beispiel.
    \item Gib die Ergebnismenge beim Werfen eines Würfels an.
    \item Welche Ereignisse sind beim Werfen einer Münze möglich?
\end{enumerate}

\subsection*{2. Laplace-Wahrscheinlichkeit}
\textbf{Formel:} $P(E) = \dfrac{\text{Anzahl der günstigen Ergebnisse}}{\text{Anzahl aller möglichen Ergebnisse}}$

\begin{enumerate}[a)]
    \item Wie groß ist die Wahrscheinlichkeit, beim Würfeln eine 5 zu werfen?
    \item Wie groß ist die Wahrscheinlichkeit, beim Würfeln eine gerade Zahl zu werfen?
    \item Wie groß ist die Wahrscheinlichkeit, beim einmaligen Ziehen einer Karte aus einem Skatspiel (32 Karten) ein Ass zu ziehen?
\end{enumerate}

\subsection*{3. Baumdiagramme und Pfadregeln}
\begin{enumerate}[a)]
    \item Zeichne ein Baumdiagramm für zweimaliges Werfen einer Münze.
    \item Berechne die Wahrscheinlichkeit, zweimal hintereinander "Kopf" zu werfen.
    \item Ein Glücksrad hat 3 gleich große Felder (rot, blau, grün). Zeichne ein Baumdiagramm für zweimaliges Drehen und berechne die Wahrscheinlichkeit, zweimal die gleiche Farbe zu erhalten.
\end{enumerate}

\subsection*{4. Kombinatorik}
\begin{enumerate}[a)]
    \item Wie viele verschiedene Möglichkeiten gibt es, die Buchstaben A, B und C nebeneinander anzuordnen?
    \item In einer Klasse sitzen 5 Schüler in einer Reihe. Wie viele Sitzordnungen sind möglich?
    \item Aus 6 Schülern werden 2 Klassensprecher gewählt. Wie viele Möglichkeiten gibt es?
\end{enumerate}

\subsection*{5. Erwartungswert}
\begin{enumerate}[a)]
    \item Ein Würfelspiel: Bei einer 6 gewinnt man 2 €, sonst verliert man 1 €. Wie hoch ist der Erwartungswert pro Spiel?
    \item Ein Glücksrad hat die Felder 1, 2, 3 (je gleich groß). Man erhält den Wert des Feldes in Euro. Wie hoch ist der Erwartungswert?
\end{enumerate}

\subsection*{6. Relative Häufigkeit und Gesetz der großen Zahlen}
\begin{enumerate}[a)]
    \item Führe ein Zufallsexperiment (z.B. 50-mal Münze werfen) durch und notiere die Ergebnisse. Berechne die relative Häufigkeit für "Kopf".
    \item Erkläre das Gesetz der großen Zahlen an einem Beispiel.
\end{enumerate}

\subsection*{7. Textaufgaben und Anwendungen}
\begin{enumerate}[a)]
    \item In einer Urne sind 4 rote, 3 blaue und 3 grüne Kugeln. Es wird eine Kugel gezogen. Wie groß ist die Wahrscheinlichkeit, eine blaue Kugel zu ziehen?
    \item In einer Lotterie gibt es 100 Lose, davon sind 5 Hauptgewinne. Wie groß ist die Gewinnwahrscheinlichkeit?
    \item Ein Würfel wird dreimal geworfen. Wie groß ist die Wahrscheinlichkeit, mindestens einmal eine 6 zu werfen?
    \item In einer Klasse sind 12 Mädchen und 8 Jungen. Es wird ein Klassensprecher ausgelost. Wie groß ist die Wahrscheinlichkeit, dass ein Mädchen gewählt wird?
\end{enumerate}

\vspace{1em}
\textbf{Tipp:} Zeichne Baumdiagramme, rechne mit Wahrscheinlichkeiten und begründe deine Lösungen ausführlich!

\newpage

\section*{Lösungen: Grundlagen der Stochastik}

\subsection*{1. Zufallsexperimente und Ereignisse}
\begin{enumerate}[a)]
    \item Beispiel: Zufallsexperiment = Würfeln; Ergebnis = "3"; Ereignis = "gerade Zahl" (also 2, 4 oder 6).
    \item $\Omega = \{1,2,3,4,5,6\}$
    \item "Kopf" oder "Zahl"
\end{enumerate}

\subsection*{2. Laplace-Wahrscheinlichkeit}
\begin{enumerate}[a)]
    \item $P(5) = \frac{1}{6}$
    \item $P(2,4,6) = \frac{3}{6} = \frac{1}{2}$
    \item $P(\text{Ass}) = \frac{4}{32} = \frac{1}{8}$
\end{enumerate}

\subsection*{3. Baumdiagramme und Pfadregeln}
\begin{enumerate}[a)]
    \item Baumdiagramm: 1. Wurf: Kopf/Zahl; 2. Wurf: Kopf/Zahl jeweils an beiden Ästen.
    \item $P(\text{Kopf, Kopf}) = \frac{1}{2} \cdot \frac{1}{2} = \frac{1}{4}$
    \item $P(\text{zweimal gleiche Farbe}) = 3 \cdot \left(\frac{1}{3} \cdot \frac{1}{3}\right) = 3 \cdot \frac{1}{9} = \frac{1}{3}$
\end{enumerate}

\subsection*{4. Kombinatorik}
\begin{enumerate}[a)]
    \item $3! = 6$ Möglichkeiten
    \item $5! = 120$ Sitzordnungen
    \item $\binom{6}{2} = 15$ Möglichkeiten
\end{enumerate}

\subsection*{5. Erwartungswert}
\begin{enumerate}[a)]
    \item $E = \frac{1}{6} \cdot 2 + \frac{5}{6} \cdot (-1) = \frac{2}{6} - \frac{5}{6} = -\frac{1}{2}$ €
    \item $E = \frac{1+2+3}{3} = 2$
\end{enumerate}

\subsection*{6. Relative Häufigkeit und Gesetz der großen Zahlen}
\begin{enumerate}[a)]
    \item Beispiel: 50-mal geworfen, 27-mal "Kopf". $f = \frac{27}{50} = 0{,}54$
    \item Die relative Häufigkeit nähert sich bei vielen Wiederholungen der theoretischen Wahrscheinlichkeit an.
\end{enumerate}

\subsection*{7. Textaufgaben und Anwendungen}
\begin{enumerate}[a)]
    \item $P(\text{blau}) = \frac{3}{10}$
    \item $P(\text{Hauptgewinn}) = \frac{5}{100} = 0{,}05$
    \item $P(\text{mind. 1 Sechs}) = 1 - (\frac{5}{6})^3 \approx 0{,}421$
    \item $P(\text{Mädchen}) = \frac{12}{20} = 0{,}6$
\end{enumerate}

[ \binom{6}{2} = \frac{6!}{2! \cdot 4!} = \frac{6 \times 5 \times 4!}{2 \times 1 \times 4!} = \frac{30}{2} = 15]