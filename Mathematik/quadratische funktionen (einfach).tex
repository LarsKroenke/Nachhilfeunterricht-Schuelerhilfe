\section*{1. Grundlagen: Parabeln und ihre Darstellung}
\begin{enumerate}
    \item Zeichne den Graphen der Funktion \( f(x) = x^2 \) in ein Koordinatensystem für \( x \) im Bereich \([-3,3]\).
    \item Beschreibe die Eigenschaften der Parabel:
    \begin{enumerate}
        \item Scheitelpunkt
        \item Symmetrie
        \item Monotonieverhalten
    \end{enumerate}
    \item Wie verändert sich der Graph von \( f(x) = x^2 \), wenn du folgende Änderungen vornimmst?  
    Skizziere die Graphen qualitativ in ein Koordinatensystem.
    \begin{enumerate}
        \item \( g(x) = x^2 + 2 \)
        \item \( h(x) = x^2 - 3 \)
        \item \( p(x) = (x - 2)^2 \)
        \item \( q(x) = -x^2 \)
    \end{enumerate}
\end{enumerate}




\section*{3. Schnittpunkte mit den Koordinatenachsen}
\begin{enumerate}
    \item Bestimme die Nullstellen der folgenden Funktionen graphisch und rechnerisch:
    \begin{enumerate}
        \item \( f(x) = x^2 - 4 \)
        \item \( g(x) = x^2 - 2x - 3 \)
    \end{enumerate}
    \item Zeichne die Parabeln aus der vorherigen Aufgabe und überprüfe deine Ergebnisse.
    \item Bestimme den Schnittpunkt mit der \( y \)-Achse für die Funktion \( f(x) = -x^2 + 3x - 2 \).
\end{enumerate}

\section*{4. Quadratische Funktionen und ihre Anwendung}
\begin{enumerate}
    \item Ein Ball wird senkrecht nach oben geworfen. Seine Höhe (in Metern) nach \( t \) Sekunden wird durch die Funktion  
    \[
    h(t) = -5t^2 + 10t + 2
    \]
    beschrieben.
    \begin{enumerate}
        \item Skizziere den Graphen der Funktion für \( t \in [0,3] \).
        \item Wann erreicht der Ball seine maximale Höhe?  
        \item Bestimme die maximale Höhe.
        \item Wann landet der Ball wieder auf dem Boden?
    \end{enumerate}
\end{enumerate}