% --- Aufgabenblatt: Rechtschreibung (10. Klasse) ---

\section*{Rechtschreibung}

\subsection*{1. Groß- und Kleinschreibung}
\textbf{A) Setze die Groß- und Kleinschreibung korrekt:}
\begin{enumerate}[a)]
    \item gestern gingen wir ins kino und sahen einen spannenden film.
    \item die lehrerin erklärte die regeln der deutschen rechtschreibung.
    \item im sommer fahren viele familien an die ostsee.
    \item er hat angst vor prüfungen.
\end{enumerate}

\subsection*{2. Getrennt- und Zusammenschreibung}
\textbf{B) Schreibe die folgenden Wortgruppen richtig:}
\begin{enumerate}[a)]
    \item rad fahren
    \item kennen lernen
    \item sitzen bleiben
    \item spazieren gehen
    \item statt finden
    \item teil nehmen
\end{enumerate}

\subsection*{3. Zeichensetzung (Kommata, Punkte, Striche)}
\textbf{D) Setze die Satzzeichen richtig:}
\begin{enumerate}[a)]
    \item Wenn es regnet gehen wir ins Museum
    \item Ich mag Bücher besonders Krimis und Romane
    \item Sie kam nach Hause machte ihre Hausaufgaben und ging schlafen
    \item Obwohl er müde war lernte er weiter
\end{enumerate}

\subsection*{4. Fremdwörter und schwierige Wörter}
\textbf{E) Schreibe die Wörter richtig:}
\begin{enumerate}[a)]
    \item Portmonä
    \item Komunikation
    \item Rythmus
    \item Akzeptans
    \item Desinterreße
\end{enumerate}

\subsection*{5. Fehlertext}
\textbf{F) Finde und verbessere die Rechtschreibfehler:}

\textit{gestern abend habe ich mit meinen freunden ein film geschaut wir haben pizza gegesen und viel gelacht am nächsten tag war ich sehr müde trotzdem bin ich pünktlich zur schule gegangen}

\subsection*{6. Diktat (Lehrkraft liest vor)}
\textbf{G) Schreibe den folgenden Text fehlerfrei auf:}

\textit{Der Sommer war ungewöhnlich heiß. Viele Menschen verbrachten ihre Freizeit am See oder im Freibad. Abends saßen sie oft lange draußen und genossen die lauen Nächte.}

\subsection*{7. Fehlertext}
\textbf{H) Finde und verbessere alle Rechtschreibfehler im folgenden Text:}

\textit{am montag morgen bin ich viel zu spät aufgestanden weil mein wecker nicht geklingelt hat. ich habe schnell meine schulsachen gepackt und bin ohne zu frühstücken aus dem haus gerannt. auf dem weg zur schule habe ich bemerkt das ich mein mathebuch vergessen habe. im untericht war ich sehr müde und konnte mich kaum konzentrieren. die lehrerin hat eine wichtige ankündigung gemacht aber ich habe sie nicht richtig verstanden. nach der schule bin ich mit meinen freunden ins cafe gegangen wir haben über die kommende klassenarbeit gesprochen und uns gegenseitig tipps gegeben. zuhause angekommen habe ich erstmal etwas gegesen und dann meine hausaufgaben gemacht. am abend habe ich noch ein bischen ferngesehen bevor ich ins bett gegangen bin.}
