% Aufgabenblatt: Mediation Englisch 12. Klasse
% Thema: Mediation und Übersetzung

\section*{Mediation}

\subsection*{Hinweise zur Mediation}
Mediation bedeutet, Informationen aus einem Text oder Gespräch sinngemäß und adressatengerecht in eine andere Sprache zu übertragen. Es geht nicht um wortwörtliche Übersetzung, sondern um sinngemäße Wiedergabe. Achte auf einen klaren, verständlichen und grammatikalisch korrekten Ausdruck.

\subsection*{Aufgabe 1: Deutsche Sätze ins Englische übersetzen}
Übersetze die folgenden deutschen Sätze sinngemäß ins Englische. Achte auf einen natürlichen Ausdruck und die richtige Zeitform.

\begin{enumerate}
    \item Die Schule plant, im nächsten Jahr einen Austausch mit einer englischen Partnerschule zu organisieren.
    \item Viele Schülerinnen und Schüler sind daran interessiert, ihre Englischkenntnisse zu verbessern.
    \item Während des Austauschs werden die Teilnehmer bei Gastfamilien wohnen und am Unterricht teilnehmen.
    \item Die Anmeldung ist bis zum 15. September möglich.
    \item Es wird erwartet, dass die Teilnehmer offen und respektvoll mit anderen Kulturen umgehen.
    \item Die Kosten für die Reise betragen etwa 400 Euro pro Person.
    \item Bei Fragen können sich die Eltern an die Organisatoren wenden.
    \item Im letzten Jahr war der Austausch ein großer Erfolg.
    \item Die Schüler haben viele neue Freundschaften geschlossen.
    \item Einige Teilnehmer möchten ihre Austauschpartner im Sommer besuchen.
\end{enumerate}

\subsection*{Aufgabe 2: Mediation – Informationen weitergeben}
Stelle dir vor, deine englische Austauschpartnerin möchte wissen, wie das deutsche Schulsystem funktioniert. Fasse die wichtigsten Informationen in 5-6 Sätzen auf Englisch zusammen.

\subsection*{Aufgabe 3: Mediation – E-Mail vermitteln}
Du erhältst folgende E-Mail von deiner englischen Freundin. Fasse die wichtigsten Informationen auf Deutsch für deine Eltern zusammen.

\begin{quote}
Hi! I am really looking forward to visiting you in Germany. I would like to know more about your school and your daily routine. Could you also tell me if there are any special rules I should be aware of? I am a bit nervous about speaking German all the time. See you soon!\newline
Best,\newline
Emily
\end{quote}

\subsection*{Aufgabe 4: Mediation – Dialog vermitteln}
Zwei Mitschüler unterhalten sich auf Deutsch über ihre Erfahrungen im Ausland. Gib die wichtigsten Punkte auf Englisch wieder (5-6 Sätze).

\begin{quote}
Anna: Ich fand es am Anfang schwierig, mich an das Essen zu gewöhnen.\newline
Max: Mir ging es ähnlich, aber die Gastfamilie war sehr freundlich.\newline
Anna: Die Schule war ganz anders als in Deutschland.\newline
Max: Ja, und der Unterricht begann viel später am Morgen.\newline
Anna: Ich habe viel über die Kultur gelernt und neue Freunde gefunden.\newline
Max: Ich würde jederzeit wieder an einem Austausch teilnehmen.
\end{quote}

\subsection*{Aufgabe 5: Eigene Mediation verfassen}
Beschreibe in 6-8 Sätzen auf Englisch, wie ein typischer Schultag in Deutschland abläuft. Gehe dabei auf Unterrichtsbeginn, Pausen, Fächer und Besonderheiten ein.

\vspace{1cm}
\textbf{Viel Erfolg!}
